% ============================================================================
% STATISTICAL ARBITRAGE: ADVANCED METHODS RESEARCH
% ============================================================================
% A comprehensive literature review and methodology guide for statistical
% arbitrage strategies given unstable cointegration relationships.
%
% Context: 100 anonymized stocks, 2016-2026 OHLCV data
% Current findings: 700+ cointegrated pairs, only 45% stable
% Transaction costs: 10 bps per trade
% ============================================================================

\section{Statistical Arbitrage: Advanced Methods Research}
\label{sec:stat_arb_advanced}

\subsection{Executive Summary}

Given our diagnostic findings---700+ cointegrated pairs with only 45\% temporal stability and lead-lag relationships concentrated at lag 0---classic pairs trading will fail. This research document surveys advanced statistical arbitrage methods designed to address these challenges.

\begin{lessonbox}
\textbf{Key Findings from This Research:}
\begin{enumerate}
    \item Classic Engle-Granger cointegration is \textbf{fundamentally flawed} for trading due to static hedge ratios
    \item \textbf{Kalman Filter hedge ratios} provide the most promising path forward
    \item \textbf{Fractional cointegration} and \textbf{regime-switching models} explain our instability
    \item \textbf{PCA-based arbitrage} (Principal Portfolios) may outperform pairwise approaches
    \item Transaction cost mitigation requires \textbf{threshold-based execution} and \textbf{position smoothing}
\end{enumerate}
\end{lessonbox}

% ============================================================================
\subsection{Part 1: Why Classic Pairs Trading Fails}
% ============================================================================

\subsubsection{The Engle-Granger Framework: Assumptions and Violations}

The Engle-Granger (1987) two-step cointegration test assumes:

\begin{mathbox}
\textbf{Engle-Granger Cointegration:}

For two I(1) series $Y_t$ and $X_t$:
\begin{equation}
    Y_t = \alpha + \beta X_t + \epsilon_t
\end{equation}

Cointegration holds if $\epsilon_t \sim I(0)$ (stationary). The test:
\begin{enumerate}
    \item Estimate $\hat{\beta}$ via OLS
    \item Test $\epsilon_t = Y_t - \hat{\beta}X_t$ for stationarity (ADF test)
\end{enumerate}
\end{mathbox}

\textbf{Why this fails for trading:}

\begin{itemize}
    \item \textbf{Static hedge ratio}: $\hat{\beta}$ is estimated once over the full sample. Our rolling tests show relationships break down 55\% of the time.
    
    \item \textbf{Parameter instability}: Gregory and Hansen (1996) show cointegration parameters experience structural breaks. Financial time series are \textbf{non-ergodic}---past distributions don't predict future distributions.
    
    \item \textbf{Spurious regression risk}: Granger and Newbold (1974) demonstrate that regressing two independent random walks yields spuriously significant $\hat{\beta}$.
    
    \item \textbf{Speed of mean reversion}: Even if cointegrated, the half-life of mean reversion may exceed our holding period.
\end{itemize}

\textbf{Literature:}
\begin{itemize}
    \item Engle, R. F., \& Granger, C. W. J. (1987). ``Co-integration and Error Correction: Representation, Estimation, and Testing.'' \textit{Econometrica}, 55(2), 251-276.
    \item Gregory, A. W., \& Hansen, B. E. (1996). ``Residual-Based Tests for Cointegration in Models with Regime Shifts.'' \textit{Journal of Econometrics}, 70(1), 99-126.
\end{itemize}

% ============================================================================
\subsubsection{Dynamic Hedge Ratios via Kalman Filter}
% ============================================================================

The Kalman Filter provides time-varying hedge ratio estimation, directly addressing our instability problem.

\begin{mathbox}
\textbf{State-Space Formulation for Dynamic Hedging:}

\textbf{Observation equation:}
\begin{equation}
    y_t = \beta_t x_t + \epsilon_t, \quad \epsilon_t \sim N(0, R)
\end{equation}

\textbf{State transition equation:}
\begin{equation}
    \beta_t = \beta_{t-1} + \eta_t, \quad \eta_t \sim N(0, Q)
\end{equation}

where:
\begin{itemize}
    \item $y_t$ = log price of asset Y
    \item $x_t$ = log price of asset X
    \item $\beta_t$ = time-varying hedge ratio
    \item $Q$ = state covariance (controls adaptation speed)
    \item $R$ = observation noise variance
\end{itemize}
\end{mathbox}

\textbf{Kalman Filter Update Equations:}

\begin{mathbox}
\textbf{Prediction step:}
\begin{align}
    \hat{\beta}_{t|t-1} &= \hat{\beta}_{t-1|t-1} \\
    P_{t|t-1} &= P_{t-1|t-1} + Q
\end{align}

\textbf{Update step:}
\begin{align}
    K_t &= \frac{P_{t|t-1} x_t^2}{P_{t|t-1} x_t^2 + R} \quad \text{(Kalman gain)} \\
    \hat{\beta}_{t|t} &= \hat{\beta}_{t|t-1} + K_t(y_t - \hat{\beta}_{t|t-1} x_t) \\
    P_{t|t} &= (1 - K_t x_t) P_{t|t-1}
\end{align}

The \textbf{spread} is then:
\begin{equation}
    z_t = y_t - \hat{\beta}_{t|t} x_t
\end{equation}
\end{mathbox}

\textbf{Tuning considerations:}
\begin{itemize}
    \item Small $Q$: Slow adaptation, more stable but may miss regime changes
    \item Large $Q$: Fast adaptation, more responsive but noisier
    \item Typical initialization: $Q/R \in [10^{-5}, 10^{-3}]$
\end{itemize}

\textbf{Advantages over static OLS:}
\begin{enumerate}
    \item Adapts to structural breaks automatically
    \item Provides uncertainty estimates ($P_{t|t}$) for position sizing
    \item Can be extended to multivariate settings (multiple assets)
\end{enumerate}

\textbf{Literature:}
\begin{itemize}
    \item Elliott, G., \& Müller, U. K. (2006). ``Efficient Tests for General Persistent Time Variation in Regression Coefficients.'' \textit{Review of Economic Studies}, 73(4), 907-940.
    \item Montana, G., et al. (2009). ``Flexible Least Squares for Temporal Data Mining and Statistical Arbitrage.'' \textit{Expert Systems with Applications}, 36, 2819-2830.
\end{itemize}

% ============================================================================
\subsubsection{Fractional Cointegration}
% ============================================================================

Our 45\% stability finding may indicate \textbf{fractional cointegration}---a weaker form where the spread is mean-reverting but has long memory.

\begin{mathbox}
\textbf{Fractional Integration:}

A process $X_t$ is $I(d)$ where $d \in (0, 1)$:
\begin{equation}
    (1-L)^d X_t = \epsilon_t
\end{equation}

where $(1-L)^d = \sum_{k=0}^{\infty} \binom{d}{k} (-L)^k$ is the fractional difference operator.

\textbf{Fractional Cointegration:}
Two $I(1)$ series are fractionally cointegrated if their linear combination is $I(d)$ with $0 < d < 1$.
\end{mathbox}

\textbf{Implications for trading:}
\begin{itemize}
    \item Mean reversion exists but is \textbf{slower} than standard cointegration
    \item Standard ADF tests have \textbf{low power} against fractional alternatives
    \item The spread can wander far from the mean before reverting
\end{itemize}

\textbf{Testing for fractional cointegration:}
\begin{enumerate}
    \item \textbf{GPH estimator} (Geweke \& Porter-Hudak, 1983): Log-periodogram regression
    \item \textbf{Local Whittle estimator}: More efficient, semi-parametric
    \item \textbf{KPSS variant}: Phillips \& Shimotsu (2004)
\end{enumerate}

\begin{mathbox}
\textbf{GPH Estimator:}

Regress log periodogram on log frequency:
\begin{equation}
    \log I(\omega_j) = a - d \cdot \log|2\sin(\omega_j/2)| + u_j
\end{equation}

where $\omega_j = 2\pi j/T$ for $j = 1, \ldots, m$ and $m = T^{0.5}$ typically.
\end{mathbox}

\textbf{Literature:}
\begin{itemize}
    \item Baillie, R. T. (1996). ``Long Memory Processes and Fractional Integration in Econometrics.'' \textit{Journal of Econometrics}, 73(1), 5-59.
    \item Cheung, Y. W., \& Lai, K. S. (1993). ``A Fractional Cointegration Analysis of Purchasing Power Parity.'' \textit{Journal of Business \& Economic Statistics}, 11(1), 103-112.
\end{itemize}

% ============================================================================
\subsubsection{Error Correction Models (ECM)}
% ============================================================================

The ECM framework provides a theoretically grounded approach to trading cointegrated pairs.

\begin{mathbox}
\textbf{Vector Error Correction Model (VECM):}

For cointegrated series $Y_t$ and $X_t$:
\begin{align}
    \Delta Y_t &= \alpha_Y (Y_{t-1} - \beta X_{t-1}) + \sum_{i=1}^{p} \gamma_i \Delta Y_{t-i} + \sum_{i=1}^{p} \delta_i \Delta X_{t-i} + \epsilon_{Y,t} \\
    \Delta X_t &= \alpha_X (Y_{t-1} - \beta X_{t-1}) + \sum_{i=1}^{p} \phi_i \Delta Y_{t-i} + \sum_{i=1}^{p} \psi_i \Delta X_{t-i} + \epsilon_{X,t}
\end{align}

where:
\begin{itemize}
    \item $\alpha_Y, \alpha_X$ = speed of adjustment coefficients
    \item $(Y_{t-1} - \beta X_{t-1})$ = error correction term (the spread)
    \item $\gamma_i, \delta_i, \phi_i, \psi_i$ = short-run dynamics
\end{itemize}
\end{mathbox}

\textbf{Trading interpretation:}
\begin{itemize}
    \item If $\alpha_Y < 0$: $Y$ adjusts back when spread is positive (mean-reverting in Y)
    \item If $|\alpha_Y| > |\alpha_X|$: $Y$ is the ``follower'', $X$ is the ``leader''
    \item Predict $\Delta Y_t$ and $\Delta X_t$ for directional bets
\end{itemize}

\textbf{Half-life of mean reversion:}
\begin{equation}
    \tau_{1/2} = \frac{\ln(2)}{-\ln(1 + \alpha)}
\end{equation}

where $\alpha$ is the combined adjustment speed. Our 10 bps transaction costs require $\tau_{1/2} < 20$ days for profitability.

% ============================================================================
\subsection{Part 2: Network-Based and Clustering Approaches}
% ============================================================================

\subsubsection{Minimum Spanning Tree (MST) for Asset Structur}

The MST reduces the correlation matrix to a tree structure, revealing market topology.

\begin{mathbox}
\textbf{MST Construction:}

\textbf{Step 1:} Convert correlation to distance:
\begin{equation}
    d_{ij} = \sqrt{2(1 - \rho_{ij})}
\end{equation}
This is the \textit{ultrametric distance} satisfying the triangle inequality.

\textbf{Step 2:} Apply Prim's or Kruskal's algorithm to find the minimum spanning tree.

\textbf{Properties:}
\begin{itemize}
    \item MST has $N-1$ edges for $N$ assets
    \item Preserves hierarchical structure
    \item Central nodes are ``systemically important''
\end{itemize}
\end{mathbox}

\textbf{Trading applications:}
\begin{enumerate}
    \item \textbf{Sector identification}: Clusters in MST correspond to sectors
    \item \textbf{Contagion risk}: Central nodes transmit shocks
    \item \textbf{Pairs selection}: Trade edges of the MST (natural pairs)
    \item \textbf{Portfolio construction}: MST-based diversification
\end{enumerate}

\textbf{Literature:}
\begin{itemize}
    \item Mantegna, R. N. (1999). ``Hierarchical Structure in Financial Markets.'' \textit{European Physical Journal B}, 11(1), 193-197.
    \item Bonanno, G., et al. (2003). ``Topology of Correlation-Based Minimal Spanning Trees in Real and Model Markets.'' \textit{Physical Review E}, 68(4), 046130.
\end{itemize}

% ============================================================================
\subsubsection{Principal Portfolios and PCA-Based Arbitrage}
% ============================================================================

PCA decomposes returns into orthogonal factors, enabling factor-neutral arbitrage.

\begin{mathbox}
\textbf{Principal Component Analysis:}

Decompose the return covariance matrix:
\begin{equation}
    \Sigma = V \Lambda V'
\end{equation}

where $V = [v_1, \ldots, v_N]$ are eigenvectors and $\Lambda = \text{diag}(\lambda_1, \ldots, \lambda_N)$ are eigenvalues.

\textbf{Principal portfolios:}
\begin{equation}
    F_k = v_k' r_t
\end{equation}

Typically, $k \leq 5$ factors explain 60-80\% of variance.
\end{mathbox}

\textbf{PCA-based arbitrage strategies:}

\begin{mathbox}
\textbf{1. Statistical Factor Arbitrage:}

Regress each asset on the first $K$ principal components:
\begin{equation}
    r_{i,t} = \alpha_i + \sum_{k=1}^{K} \beta_{i,k} F_{k,t} + \epsilon_{i,t}
\end{equation}

\textbf{Trading signal:} Cumulative residuals $e_{i,t} = \sum_{s=1}^{t} \epsilon_{i,s}$

\textbf{Strategy:}
\begin{itemize}
    \item Long assets with most negative cumulative residuals
    \item Short assets with most positive cumulative residuals
    \item Positions are factor-neutral by construction
\end{itemize}
\end{mathbox}

\begin{mathbox}
\textbf{2. Eigenportfolio Mean Reversion:}

Higher-order principal portfolios (small $\lambda_k$) are less persistent:
\begin{equation}
    F_k = v_k' r_t
\end{equation}

For $k > K$, $F_k$ represents ``noise'' that mean-reverts.

\textbf{Strategy:} Trade mean reversion in higher-order PCs.
\end{mathbox}

\textbf{Advantages:}
\begin{itemize}
    \item Automatically hedges systematic risk
    \item Reduces dimensionality from $N$ assets to $K$ factors
    \item More stable than pairwise relationships
\end{itemize}

\textbf{Challenges:}
\begin{itemize}
    \item Principal components are not stationary
    \item Rolling PCA can cause sign flips
    \item Hard to interpret economically
\end{itemize}

\textbf{Literature:}
\begin{itemize}
    \item Avellaneda, M., \& Lee, J. H. (2010). ``Statistical Arbitrage in the U.S. Equities Market.'' \textit{Quantitative Finance}, 10(7), 761-782.
    \item Connor, G., \& Korajczyk, R. A. (1988). ``Risk and Return in an Equilibrium APT.'' \textit{Journal of Financial Economics}, 21(2), 255-289.
\end{itemize}

% ============================================================================
\subsubsection{Correlation Clustering for Portfolio Construction}
% ============================================================================

\begin{mathbox}
\textbf{Hierarchical Risk Parity (HRP):}

A clustering-based portfolio allocation method:

\textbf{Step 1:} Hierarchical clustering on correlation distance
\begin{equation}
    d_{ij} = \sqrt{\frac{1}{2}(1 - \rho_{ij})}
\end{equation}

\textbf{Step 2:} Quasi-diagonalize the covariance matrix by reordering assets according to cluster structure

\textbf{Step 3:} Recursive bisection allocation:
\begin{equation}
    w_i \propto \frac{1}{\text{Var}(\text{cluster containing } i)}
\end{equation}
\end{mathbox}

\textbf{Advantages over mean-variance optimization:}
\begin{itemize}
    \item Does not require covariance matrix inversion (ill-conditioned)
    \item More stable out-of-sample
    \item Respects cluster structure
\end{itemize}

\textbf{Literature:}
\begin{itemize}
    \item López de Prado, M. (2016). ``Building Diversified Portfolios that Outperform Out-of-Sample.'' \textit{Journal of Portfolio Management}, 42(4), 59-69.
\end{itemize}

% ============================================================================
\subsection{Part 3: Lead-Lag Relationships}
% ============================================================================

\subsubsection{Why We Found No Lead-Lag at Daily Frequency}

Our analysis showed all top pairs have maximum cross-correlation at lag 0. This is consistent with market efficiency at daily frequency.

\textbf{Explanations:}
\begin{enumerate}
    \item \textbf{Information incorporation}: News is priced within minutes
    \item \textbf{Arbitrageurs}: Lead-lag relationships are arbitraged away
    \item \textbf{Wrong frequency}: Lead-lag may exist at intraday level
\end{enumerate}

\subsubsection{Cross-Correlation Analysis}

\begin{mathbox}
\textbf{Cross-Correlation Function:}
\begin{equation}
    \rho_{XY}(\tau) = \frac{\text{Cov}(X_t, Y_{t+\tau})}{\sigma_X \sigma_Y}
\end{equation}

\textbf{Interpretation:}
\begin{itemize}
    \item $\rho_{XY}(\tau) > 0$ for $\tau > 0$: $X$ leads $Y$
    \item Peak at $\tau = 0$: Synchronous movement
    \item Asymmetry: $\rho_{XY}(\tau) \neq \rho_{XY}(-\tau)$ indicates directional lead-lag
\end{itemize}
\end{mathbox}

\subsubsection{Granger Causality}

A more rigorous test for predictive relationships.

\begin{mathbox}
\textbf{Granger Causality Test:}

$X$ Granger-causes $Y$ if:
\begin{equation}
    E[Y_t | Y_{t-1}, \ldots, Y_{t-p}, X_{t-1}, \ldots, X_{t-p}] \neq E[Y_t | Y_{t-1}, \ldots, Y_{t-p}]
\end{equation}

\textbf{Implementation:} Compare two VAR models:
\begin{align}
    \text{Restricted:} \quad Y_t &= \sum_{i=1}^{p} \alpha_i Y_{t-i} + \epsilon_t \\
    \text{Unrestricted:} \quad Y_t &= \sum_{i=1}^{p} \alpha_i Y_{t-i} + \sum_{i=1}^{p} \beta_i X_{t-i} + \epsilon_t
\end{align}

\textbf{F-test:} $H_0$: $\beta_1 = \cdots = \beta_p = 0$ (no Granger causality)
\end{mathbox}

\textbf{Trading application:}
\begin{itemize}
    \item If $X$ Granger-causes $Y$, predict $Y_{t+1}$ using $X_t$
    \item Use as a feature in ML models
    \item Identify ``information leaders'' in the market
\end{itemize}

\subsubsection{Information Flow Metrics}

\begin{mathbox}
\textbf{Transfer Entropy:}

A non-linear measure of directional information flow:
\begin{equation}
    T_{X \to Y} = H(Y_t | Y_{t-1}^{(k)}) - H(Y_t | Y_{t-1}^{(k)}, X_{t-1}^{(l)})
\end{equation}

where $H(\cdot)$ is entropy and superscripts indicate lag order.

\textbf{Interpretation:}
\begin{itemize}
    \item $T_{X \to Y} > 0$: $X$ provides information about $Y$
    \item $T_{X \to Y} > T_{Y \to X}$: Net information flow from $X$ to $Y$
\end{itemize}
\end{mathbox}

\textbf{Advantages over Granger causality:}
\begin{itemize}
    \item Captures non-linear relationships
    \item Model-free (no VAR assumption)
    \item Asymmetric by construction
\end{itemize}

\textbf{Literature:}
\begin{itemize}
    \item Schreiber, T. (2000). ``Measuring Information Transfer.'' \textit{Physical Review Letters}, 85(2), 461.
    \item Kwon, O., \& Yang, J. S. (2008). ``Information Flow between Stock Indices.'' \textit{Europhysics Letters}, 82(6), 68003.
\end{itemize}

% ============================================================================
\subsection{Part 4: Regime-Adaptive Statistical Arbitrage}
% ============================================================================

\subsubsection{Integration with HMM Regime Detection}

Our existing HMM identifies bull/bear/sideways regimes. This should inform stat arb strategy.

\begin{mathbox}
\textbf{Regime-Conditional Correlations:}

Correlations change across regimes:
\begin{equation}
    \rho_{ij}^{(s)} = \text{Corr}(r_i, r_j | S_t = s)
\end{equation}

where $S_t \in \{1, 2, 3\}$ (bull, bear, sideways).

\textbf{Empirical fact:} During crises, correlations increase (``correlation breakdown''):
\begin{equation}
    \rho_{ij}^{(\text{bear})} > \rho_{ij}^{(\text{bull})}
\end{equation}
\end{mathbox}

\textbf{Implications for stat arb:}
\begin{enumerate}
    \item \textbf{Pairs selection}: Different optimal pairs per regime
    \item \textbf{Hedge ratios}: Recalibrate when regime changes
    \item \textbf{Position sizing}: Reduce exposure in high-correlation regimes
    \item \textbf{Exit rules}: Widen thresholds in volatile regimes
\end{enumerate}

\subsubsection{Markov-Switching Cointegration}

\begin{mathbox}
\textbf{Markov-Switching VECM:}

Extend the ECM to regime-dependent parameters:
\begin{equation}
    \Delta Y_t = \alpha_{S_t} (Y_{t-1} - \beta_{S_t} X_{t-1}) + \epsilon_t
\end{equation}

where $S_t$ follows a Markov chain:
\begin{equation}
    P(S_t = j | S_{t-1} = i) = p_{ij}
\end{equation}

\textbf{Estimation:} Hamilton (1989) filter or Bayesian MCMC
\end{mathbox}

\textbf{This explains our instability findings:}
\begin{itemize}
    \item Cointegration holds in regime 1 ($\alpha_1 < 0$) but not regime 2 ($\alpha_2 \approx 0$)
    \item The 45\% stability = fraction of time in regime 1
    \item Solution: Only trade when HMM indicates ``stable'' regime
\end{itemize}

\subsubsection{Regime-Specific Strategies}

\begin{table}[H]
\centering
\begin{tabular}{@{}llll@{}}
\toprule
\textbf{Regime} & \textbf{Characteristics} & \textbf{Stat Arb Strategy} & \textbf{Position Size} \\
\midrule
Bull & Low vol, low corr & Aggressive pairs trading & Full \\
Bear & High vol, high corr & Reduce/exit positions & 25\% or none \\
Sideways & Moderate & Selective pairs trading & 50\% \\
\bottomrule
\end{tabular}
\caption{Regime-specific statistical arbitrage adjustments}
\end{table}

\textbf{Literature:}
\begin{itemize}
    \item Hamilton, J. D. (1989). ``A New Approach to the Economic Analysis of Nonstationary Time Series and the Business Cycle.'' \textit{Econometrica}, 57(2), 357-384.
    \item Ang, A., \& Bekaert, G. (2002). ``Regime Switches in Interest Rates.'' \textit{Journal of Business \& Economic Statistics}, 20(2), 163-182.
\end{itemize}

% ============================================================================
\subsection{Part 5: Mathematical Formulations Summary}
% ============================================================================

\subsubsection{Complete Trading System Equations}

\begin{mathbox}
\textbf{1. Kalman Filter Hedge Ratio (Recommended):}
\begin{align}
    \text{Prediction:} \quad \hat{\beta}_{t|t-1} &= \hat{\beta}_{t-1} \\
    P_{t|t-1} &= P_{t-1} + Q \\
    \text{Innovation:} \quad \nu_t &= y_t - \hat{\beta}_{t|t-1} x_t \\
    \text{Update:} \quad K_t &= \frac{P_{t|t-1} x_t}{P_{t|t-1} x_t^2 + R} \\
    \hat{\beta}_t &= \hat{\beta}_{t|t-1} + K_t \nu_t \\
    P_t &= (1 - K_t x_t) P_{t|t-1}
\end{align}
\end{mathbox}

\begin{mathbox}
\textbf{2. Spread Z-Score with Adaptive Parameters:}
\begin{equation}
    z_t = \frac{y_t - \hat{\beta}_t x_t - \mu_t^{(\text{EMA})}}{\sigma_t^{(\text{EWMA})}}
\end{equation}
where $\mu_t^{(\text{EMA})}$ and $\sigma_t^{(\text{EWMA})}$ use exponential smoothing.
\end{mathbox}

\begin{mathbox}
\textbf{3. Position with Transaction Cost Threshold:}
\begin{equation}
    w_t = \begin{cases}
        -1 & \text{if } z_t > z_{\text{entry}}^{+} \text{ and } w_{t-1} \leq 0 \\
        +1 & \text{if } z_t < z_{\text{entry}}^{-} \text{ and } w_{t-1} \geq 0 \\
        0 & \text{if } |z_t| < z_{\text{exit}} \\
        w_{t-1} & \text{otherwise (no trade)}
    \end{cases}
\end{equation}
where $z_{\text{entry}}^{\pm} = \pm(2 + \delta)$ with $\delta$ accounting for transaction costs.
\end{mathbox}

\begin{mathbox}
\textbf{4. Regime-Weighted Signal:}
\begin{equation}
    \tilde{z}_t = \lambda_{S_t} \cdot z_t
\end{equation}
where $\lambda_s$ is the regime-specific weight from HMM:
\begin{itemize}
    \item $\lambda_{\text{bull}} = 1.0$
    \item $\lambda_{\text{sideways}} = 0.5$
    \item $\lambda_{\text{bear}} = 0.0$ (no trading)
\end{itemize}
\end{mathbox}

\subsubsection{Parameter Estimation Challenges}

\begin{table}[H]
\centering
\begin{tabular}{@{}lll@{}}
\toprule
\textbf{Parameter} & \textbf{Challenge} & \textbf{Solution} \\
\midrule
Kalman $Q$, $R$ & Unknown, affects adaptation & EM algorithm or cross-validation \\
Entry threshold $z_{\text{entry}}$ & Tradeoff: trades vs. costs & Optimize on IS, expect decay \\
Half-life $\tau_{1/2}$ & Changes over time & Rolling estimation \\
HMM transitions $p_{ij}$ & Estimation error & Bayesian with priors \\
\bottomrule
\end{tabular}
\caption{Parameter estimation challenges and solutions}
\end{table}

\subsubsection{Transaction Cost Considerations}

\begin{mathbox}
\textbf{Break-Even Analysis:}

For a round-trip trade costing $2c$ (entry + exit), required profit:
\begin{equation}
    \text{Required } \Delta z > \frac{2c}{\sigma_{\text{spread}}}
\end{equation}

With 10 bps costs and typical spread $\sigma = 2\%$:
\begin{equation}
    \Delta z_{\text{min}} = \frac{0.002}{0.02} = 0.1
\end{equation}

\textbf{Implication:} Entry at $z = 2$, exit at $z = 0.5$ gives $\Delta z = 1.5$, which is profitable.

But if half-life is 30 days and we hold for 30 days:
\begin{equation}
    \text{Expected profit} = 0.5 \times 1.5 \times 0.02 = 1.5\%
\end{equation}
Net of 20 bps costs: $1.5\% - 0.2\% = 1.3\%$ per trade. Annualized: $1.3\% \times 12 = 15.6\%$.
\end{mathbox}

% ============================================================================
\subsection{Part 6: Implementation Recommendations}
% ============================================================================

\subsubsection{Ranked by Feasibility}

\begin{table}[H]
\centering
\begin{tabular}{@{}clcccc@{}}
\toprule
\textbf{Rank} & \textbf{Method} & \textbf{Feasibility} & \textbf{Expected $\Delta$SR} & \textbf{Compute} & \textbf{Data Req.} \\
\midrule
1 & Kalman Filter Hedge Ratios & \textcolor{green!70!black}{High} & +0.1-0.3 & Low & Existing \\
2 & PCA Statistical Arbitrage & \textcolor{green!70!black}{High} & +0.2-0.4 & Low & Existing \\
3 & Regime-Conditional Trading & \textcolor{green!70!black}{High} & +0.1-0.2 & Low & Existing \\
4 & Hierarchical Risk Parity & \textcolor{orange}{Medium} & +0.1 & Low & Existing \\
5 & Fractional Cointegration & \textcolor{orange}{Medium} & +0.0-0.1 & Medium & Existing \\
6 & Transfer Entropy Lead-Lag & \textcolor{red}{Low} & +0.0-0.1 & High & Intraday \\
7 & Markov-Switching VECM & \textcolor{red}{Low} & +0.1-0.2 & High & Existing \\
\bottomrule
\end{tabular}
\caption{Implementation recommendations ranked by feasibility}
\end{table}

\subsubsection{Recommended Implementation Path}

\begin{enumerate}
    \item \textbf{Immediate (already have infrastructure):}
    \begin{itemize}
        \item Integrate Kalman Filter hedge ratios into existing pairs framework
        \item Add regime-conditional weights from HMM to stat arb signals
        \item Test PCA residual mean reversion as an additional alpha source
    \end{itemize}
    
    \item \textbf{Short-term (1-2 weeks):}
    \begin{itemize}
        \item Implement position smoothing for turnover reduction
        \item Add transaction cost threshold to entry/exit rules
        \item Validate on OOS holdout
    \end{itemize}
    
    \item \textbf{Medium-term (optional):}
    \begin{itemize}
        \item Test fractional cointegration for pair selection
        \item Explore Granger causality features for ML model
        \item Consider multi-asset spreads (baskets vs. pairs)
    \end{itemize}
\end{enumerate}

\subsubsection{Realistic Expectations}

\begin{itemize}
    \item \textbf{Current OOS Sharpe: 2.81} (Kalman + LightGBM ensemble)
    \item \textbf{Stat arb overlay potential: +0.1 to +0.3 Sharpe} (if done correctly)
    \item \textbf{Risks:}
    \begin{itemize}
        \item Stat arb may be correlated with existing signals (no diversification)
        \item Transaction costs eat into gains
        \item Regime detection errors → false entries
    \end{itemize}
    \item \textbf{Key success factor:} Keep turnover low ($<20\%$ monthly)
\end{itemize}

\begin{insightbox}
\textbf{Final Recommendation:}

Given that your current system achieves OOS Sharpe of 2.81, the marginal benefit of adding stat arb is likely \textbf{small}. The most promising approach is:

\begin{enumerate}
    \item Use \textbf{PCA residuals} as a feature in your LightGBM model (not as a standalone strategy)
    \item Apply \textbf{regime-conditional weighting} to reduce drawdowns during bear markets
    \item Use \textbf{Kalman Filter spreads} only for the most stable pairs (top 10-20) as a diversifying overlay
\end{enumerate}

The stat arb analysis is most valuable as \textbf{insight into market structure} rather than a standalone profit center.
\end{insightbox}

% ============================================================================
\subsection{References}
% ============================================================================

\begin{enumerate}
    \item Avellaneda, M., \& Lee, J. H. (2010). ``Statistical Arbitrage in the U.S. Equities Market.'' \textit{Quantitative Finance}, 10(7), 761-782.
    \item Baillie, R. T. (1996). ``Long Memory Processes and Fractional Integration in Econometrics.'' \textit{Journal of Econometrics}, 73(1), 5-59.
    \item Bonanno, G., et al. (2003). ``Topology of Correlation-Based Minimal Spanning Trees.'' \textit{Physical Review E}, 68(4), 046130.
    \item Engle, R. F., \& Granger, C. W. J. (1987). ``Co-integration and Error Correction.'' \textit{Econometrica}, 55(2), 251-276.
    \item Gregory, A. W., \& Hansen, B. E. (1996). ``Residual-Based Tests for Cointegration with Regime Shifts.'' \textit{Journal of Econometrics}, 70(1), 99-126.
    \item Hamilton, J. D. (1989). ``A New Approach to Nonstationary Time Series.'' \textit{Econometrica}, 57(2), 357-384.
    \item López de Prado, M. (2016). ``Building Diversified Portfolios that Outperform Out-of-Sample.'' \textit{Journal of Portfolio Management}, 42(4), 59-69.
    \item Mantegna, R. N. (1999). ``Hierarchical Structure in Financial Markets.'' \textit{European Physical Journal B}, 11(1), 193-197.
    \item Montana, G., et al. (2009). ``Flexible Least Squares for Statistical Arbitrage.'' \textit{Expert Systems with Applications}, 36, 2819-2830.
    \item Schreiber, T. (2000). ``Measuring Information Transfer.'' \textit{Physical Review Letters}, 85(2), 461.
\end{enumerate}

